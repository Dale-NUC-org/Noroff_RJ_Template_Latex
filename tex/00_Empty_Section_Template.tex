% This is a section template for reflective journal


% Remove or comment out this whole environment when using this template.
{\begin{center}
    \textcolor{red}{\Huge{THIS IS AN EMPTY SECTION TEMPLATE}}

    \begin{tabular}{r @{: } p{80mm}}
        {\textcolor{blue}{Blue text}} &  Help text. Short description of how to use an environment or a section in the template.\\
        <<{\emph{text placeholder}}>> & Text between angle brackets are placeholders. Indicates a string to be replaced with input text.\\
        Normal text/black fonts & Real note/example text, how the section is intended to be used.
    \end{tabular}

\end{center}


% ----> Header informaton section - Start
\section{Lesson <<{\emph{number}}>>: <<{\emph{Lesson topic}}>>}

{\textcolor{blue}{Replace the double angle brackets with the lesson number and name.}}

Original URL: <<{\emph{link to the original blog post on the Moodle blog}}>>

{\textcolor{blue}{Replace the double angle brackets with the URL to the reflective journal blog.}}
% <---- Header informaton - End


% ----> Main reflective subsection - Start
% ----| Write reflective thoughts on the topic in general. |----
\subsection{Reflection on the days lecture and tutorial}

{\textcolor{blue}{In the list below, specify lesson topics and goals. Describe how the topics will be approached.}}

{\bfseries{Lesson/Topic Objectives \& Study Plan:}}
\begin{itemize}
    \item Topic 1 - 1 day lecture
    \item Topic 2 - 2 day lecture + lab and tests
    \item Topic 3 - 2 day lecture
\end{itemize}

A groups project covering all topics to be submitted by the end of lecture}

{\textcolor{blue}{Add critical reflective thoughts about your learning experiences. Delete this text}}


% <---- Main reflective subsection - End

% ----> Main reflective subsection - Start
% ----| Write reflective thoughts on a specific reflective topic. |----
\subsection{Reflection Topics}

{\textcolor{blue}{When there are guided topics add them as sub-headings and include your critical discussion of the topic(s)}}

<<{\emph{\blindtext[1]}}>> % Example text, comment out or remove this line.

<<{\emph{\blindtext[1]}}>> % Example text, comment out or remove this line.

\subsubsection{Provided topic 1}

{\textcolor{blue}{topic 1 example reflective discussion}}

<<{\emph{\blindtext[1]}}>> % Example text, comment out or remove this line.

<<{\emph{\blindtext[1]}}>> % Example text, comment out or remove this line.

\subsubsection{Provided topic 2}

{\textcolor{blue}{topic 2 example reflective discussion}}

% <---- Main reflective subsection - End



% ----> Key Take-Away subsection - Start
% ----| Write Itemized notes regarding the lesson topic. |----
\subsection{Key Take-Away}

\begin{table}[H]
    \begin{tabular}{p {20mm} @{: } p{80mm}}
        Lesson date & {\emph{<<yyyy mm dd>>}} \\
        Date taken & {\emph{<<yyyy mm dd>>}} \\
        Revisited & <<{\emph{comment or yyyy mm dd}}>> \\
    \end{tabular}
\end{table}

{\textcolor{blue}{This subsection outlines key information from the day's lesson in bullet points.}}

\begin{enumerate}\itshape
    \item <<Main Item 1
    \begin{enumerate}
        \item sub item 1 1
        \item sub item 1 2
        \item sub item 1 3
        \item sub item 1 4
        \item sub item 1 5
        \item sub item 1 6
    \end{enumerate}
    \item Main Item 2
    \begin{enumerate}
        \item sub item 2 1
        \item sub item 2 2
        \item sub item 2 3
        \item sub item 2 4
    \end{enumerate}
    \item Main Item 3
    \item ..
    \item .>>
\end{enumerate}


\subsection{Lessons Learned}

{\textcolor{blue}{This subsection summarizes the day's lesson topic.}}

{\bfseries{Bold heading}}

<<{\emph{\blindtext[2]}}>>


{\bfseries{Glossary:}}

{\textcolor{blue}{Expand and elaborate on new words, expressions and topic terminology with a glossary list.}}

\begin{table}[H]\label{tab:glossary}
    \begin{tabular}{p{40mm} | p{120mm}}
        {\bfseries{Key Word/Expression}} & {\bfseries{Elaboration/Comment}}\\ \hline
        <<{\emph{Input}}>> & <<{\emph{Input}}>>\\ \hline
    \end{tabular}
\end{table}

% <---- Lessons Learned subsection - End



% ----> Action-Point subsection - Start
% ----| To-do list to complete the day's lesson topic. |----
\subsection{Action Points - Further Reading/Enquiry}

{\textcolor{blue}{To-do list relevant to complete/expand on the lesson topic.}}

\begin{adjustbox}{max width=\textwidth}
    \begin{tabular}{c|l|l|c|l}
        {\bfseries{Act Pnt}} & {\bfseries{To-do description}} & {\bfseries{Assigned to}} & {\bfseries{Target date}} & {\bfseries{Comment/Status}} \\
        \hline
        1 & <<{\emph{Task name/description}}>> & <<{\emph{Task owner}}>> & <<{\emph{Deadline yyyy-mmm-dd}}>> & <<{\emph{Comments or status}}>> \\ \hline
        2 & <<{\emph{Task name/description}}>> & <<{\emph{Task owner}}>> & <<{\emph{Deadline yyyy-mmm-dd}}>> & <<{\emph{Comments or status}}>> \\ \hline
    \end{tabular}
\end{adjustbox}

% <---- Action-Point subsection - End



% ----> Other source materials subsection - Start
% ----| A list of materials not directly provided via the course books/distributed by lecturers for the day's lesson topic. |----
\subsection{Other source materials} 

{\textcolor{blue}{External course related material. Insert book title, ISBN, URL etc. }}

\begin{table}[H]\label{tab:sources}
    \begin{tabular}{p{10mm} | p {47mm} | p{100mm}}
        {\bfseries{Item num.}} & {\bfseries{Resource Type}} & {\bfseries{Source description, Book title, URL, etc.}}\\
        \hline
        1 & <<{\emph{Input}}>> & <<{\emph{Input}}>>\\ \hline
    \end{tabular}
\end{table}

% <---- Other source materials subsection - end

% ----> Issues Noted and Area of Improvements subsection - Start
% ----| A list of items, after-thought about issues, difficulties (on the subject itself, materials, tools, etc.) working with the day's lesson topic. |----
\subsection{Issues/Solutions Noted and Area of Improvements}

{\textcolor{blue}{Issues noted; what has failed with regards to tools, the lecture itself, missing course material, wrong information or solution in the course assignment etc. Any area of improvement with regards to how previous task has been managed, how to more efficiently proceed with the course personally. Suggestions for improvement suggestions to course leader.}}

\begin{table}[H]\label{tab:issuensolution}
    \begin{tabular}{p{10mm} | p {47mm} | p{100mm}}
        {\bfseries{Item num.}} & {\bfseries{Issue description / Area of Improvement}} & Comments / Solution \\ \hline
        1 & <<{\emph{input}}>> & <<{\emph{input}}>>\\ \hline
    \end{tabular}
\end{table}

% <---- Issues Noted and Area of Improvements subsection - End