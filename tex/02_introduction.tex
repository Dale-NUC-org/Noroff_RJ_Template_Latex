\section{Introduction}

% Remove or comment out this whole environment when using this template.
{\begin{center}

\end{center}


{\huge{Course: UC1NPR052 Network Principles - 2020/2021}}

Moodle URL: \url{https://lms.noroff.no/course/view.php?id=557}

\subsection{Key dates}

<<<<<<< HEAD
\begin{tabular}{r @{ : } c}
    Duration: & 6 weeks\\
    Start & 1900-01-01\\
    End & 1900-01-01\\
    Formative assessment & TBD\\
    Assessment 1 - Submission date & TBD\\
    Assessment 2 - Submission date & TBD\\
=======
\begin{tabular}{r @{: } c }
    Duration: & 3 weeks\\
    Start & 2021-02-22\\
    End & 2021-03-05\\
    Formative assessment & TBD\\
    Assessment 1 - 20\% of course mark - Online Test date & TBD\\
    Assessment 2 - 80\% of course mark - Term Paper date & TBD\\
>>>>>>> 6cc4b9a230d14685a1b1a970add5c67fdf536d1b
\end{tabular}

\subsection{Course Tutors}

<<<<<<< HEAD
\begin{tabular}{r @{ : } l}
    Course leader & Arthur Dent\\
    Course lecturer & Ford Prefect\\
    Course tutor & John Crichton\\
    course tutor & Aeryn Sun\\
    Support tutor & Ka D'Argo\\
    Support tutor & Chiana\\
    Support tutor& Rygel\\
    Support tutor & Pa'u Zotoh Zhaan\\
=======
\begin{tabular}{r @{: } l}
    Course leader & Piet Delport\\
    Course lecturer & Ruan Koen\\
>>>>>>> 6cc4b9a230d14685a1b1a970add5c67fdf536d1b
\end{tabular}

\subsection{Study goals}

{\bfseries{Excerpt from course description:}}
The course will equip students with practical knowledge of general network theory. In particular it will address network structures and topology and explore key protocols. The course will also provide the students with knowledge of how to implement networks within business.


Sets of competences, expressing what the student will know, understand or be able to
do after completion of a process of learning, and products of this process.) Students who have successfully completed this course will have gained the following:

{\bfseries{Course Objectives and Learning outcomes:}}
\begin{itemize}
    \item Course Objectives
        \begin{itemize}
            \item The student will be able to assess and construct appropriate network configurations.
        \end{itemize}
    \item Learning outcomes of the course
    \begin{itemize}
        \item Students should have knowledge of:
            \begin{itemize}
                \item General theoretical network models
                \item Network devices
                \item Forms of network communication
            \end{itemize}
        \item Students will be able to demonstrate the following skills:
            \begin{itemize}
                \item To understand how the interaction between network components enable communication via local and global networks
                \item To implement and configure network devices
                \item To construct and configure a network and subnet
                \item To reflect upon learning and network skills development
            \end{itemize}
        \item General Competence: The students will have developed or strengthened attitudes in
        relation to
            \begin{itemize}
                \item Be aware of issues of reliability and responsibility to users
                \item Be aware of the need for ‘fit for purpose’ networks
            \end{itemize}
    \end{itemize}
\end{itemize}

{\bfseries{Misc. key course Info:}}
\begin{table}[H]
    \begin{tabular}{r @{: } l }
        ECTS credits & 05.0 \\
        Estimated student workload in hours & Nominally 125 student hours. Full time study (Approximately 40 hours pr/wk). \\
    \end{tabular}
\end{table}

\begin{table}[H]
    \begin{tabular}{r @{: } l }
        Delivery Pattern & \\
        Lecture / Tutorial / Supported Study & 36 hours \\
        Group / Individual Self Study & 56 hours \\
        Assessment time & 33 hours \\
    \end{tabular}
\end{table}
