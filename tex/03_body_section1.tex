% This is a section template for reflective journal


{\begin{center}
    \textcolor{red}{\Huge{THIS IS A SAMPLE PAGE}}
\end{center}


% ----> Header informaton section - Start
\section{Lesson 1:Introduction} % Replace the double angle brackets with the lesson number and name.

Original URL: << link to the original blog post on the Moodle blog>> % Replace the double angle brackets with the URL to the reflective journal blog.
% <---- Header informaton - End

% ----> Main reflective subsection - Start
% ----| Write reflective thoughts on the topic in general. |----
\subsection{Reflection on the days lecture and tutorial}

Lesson 1 was an introduction on the subject of {\emph{Programming Databases}}. It gave a nice overview about the subject, how it is layed out, and perhaps most importantly the study goal. Compared to the 3 other subjects taken since the start of this course, it is the first subject where the subject was clearly outlined along with the goals.


There was a statement, see quote on page \pageref{quote} from Prof. Johan Van Niekerk, which is important to keep in the back of mind. It should perhaps be pinned to the wall as a reminder of a pitfall to be cognizant of when surmounting challenging study phases. A reminder to wisely allocate the effort exerted, and lower the level of pondering on the vastness of relevant topics, but stay in focus inside the subject domain, at hand.


The statement resonated with me personally, since I regard lack of focus and wasted effort as one culprit of my struggle to keep up on course materials and assessments. I find it very easy to veer of on a tangent and wander away from the study material. For example, making search queries and delving into statistics, while addressing probability in discrete math.


% ----> Main reflective subsection - End


% ----> Main reflective subsection - Start
% ----| Write reflective thoughts on a specific reflective topic. |----
\subsection{Reflection Topics}

None applicable for this lesson.

% ----> Main reflective subsection - End



% ----> Main reflective subsection - Start
% ----| Write Itemized notes regarding the lesson topic. |----
\subsection{Key Take-Away}

This subsection outlines key information from the day's lesson in bullet points.


\begin{itemize}
    \item Working with database (SQLite)
        \begin{itemize}
            \item Acquire fundamental skill about working with databases
            \item How to design as simple normalized database
            \item Understand database storage and data structure
            \item Understand database Normalization
            \item Be able to query and interface with databases
            \item How to script and automate database connection, mangagement and datamining
            \item Automate data manipulation and analysis, generating reports and statitics etc on data in databases, dataframes etc.
            \item Understanding and being able to manage and  work with databases is therefore key to the field of CyberSecurity.
        \end{itemize}
\end{itemize}


Course: UC1PR2101 - Programming Databases


\begin{enumerate}
    \item New lesson structure.
        \begin{enumerate}
            \item The course is layed out to be taken with a more individual approach, akin to remote studies. More preparation are expected prior to lecture sessions.
            \item Lessons are broken up into smaller topics.
            \item Reflective Journals are not mandated. 20\% of the mark will not be allocated to Reflective Journal submition.
            \item Quizes will be smaller and with a formative purpose. There will be a practice Quizes.
            \item Overall reduced number of submition for assessment. Course grades will be based on 1 or 2 larger assessments, instead of many smaller assessments.
            \item Course assessment targets, along with target dates to be posted soon.
        \end{enumerate}
    \item New Lecture structure.
        \begin{enumerate}
            \item Students are expected to engage with study material at least 1 day ahead.
            \item Students are expected to be more prepared for each lecture topics.
            \item Referenced resources are not "mandatory", students must choose what materials are applicable and important.
        \end{enumerate}
    \item Tools and applications
        \begin{enumerate}
            \item SQLite
            \item Python
            \item PANDAS(?)
        \end{enumerate}
\end{enumerate}

Learning Databases itself is a comprehensive part of software engineering and software development, which cannot be condensed in a 6 week course.

\begin{displayquote}\label{quote}
    {\emph{We are not software developers. Our purpose is to learn enough to be able to understand enough to know what we are looking at when we are working with someone elses (database) design templates...\\}}
    {\ttfamily{
        - Prof. Johan Van Niekerk}}
\end{displayquote}

Analogous to learning enough foriegn language; One is not expected to be a fluent speaker. But know enough, to converse and to be able to accomplish a specific goal. Deeper knowledge are obtained along the way, whil fluency and comes through effort over time. Basically; not everyone who drives a car are mechanics.

\subsubsection{Lessons Learned}

This subsection summarizes the day's lesson topic.


\begin{itemize}
    \item Why databases (in relation to CyberSecurity)?
        \begin{itemize}
            \item Acquire fundamental skills about the purpose of databases
            \item Understand how databases are key to modern data and information infrasctructure
            \item Get an overview of the majority of todays transactional databases today and their use of relational database
            \item Understand how systems and data breach are on the database connectivity and transactional level
        \end{itemize}
    \item What is a database?
        \begin{itemize}
            \item Acquire fundamental skills about what a database is
            \item What databases are used for
            \item What types of databases are in use
        \end{itemize}
    \item Where does database fit into the ecosystem of "data"?
        \begin{itemize}
            \item Be able to identify different ways of storing, structuring and organizing data.
        \end{itemize}
\end{itemize}


{\bfseries{Databases}}



Databases organize data/information by following examples.


\begin{itemize}
    \item Catagorization
    \item Quantify
    \item Itemization
    \item Relation etc.
\end{itemize}


Database systems aims to resolve some data storage issues such as problemetic {\emph{Data redundancy/duplication}}:
\begin{itemize}
    \item Storage, takes space
    \item Overhead, when updating
    \item Itegrity, data consistency
\end{itemize}


{\bfseries{Glossary:}}


% \begin{tabular}{@{}c|c @{}}
\begin{tabular}{p{20mm} | p{120mm}}
    {\bfseries{Key Word}} & {\bfseries{Elaboration/Comment}}\\ \hline
    Database & A logical way to organize, store, label and describe relationships of data.\\ \hline
    Third Normal Form & Relational databases. A database schema design see "Other source material" table in section \ref{subsec:source}. Ensures update and insert integrity to the database.\\ \hline
    (Working in) Disconnected mode & A safe way to work with data in Databases, to avoid data curruption or data integrity error. Such curruption or error can occur when multiple connections are made and edits the same data at the same time. Tracking which changes, by which connections, is the most recent and valid change will be difficult. Working in "Disconnected Mode" will remedy this issue.\\ \hline
    SQL & {\bfseries{S}}tructured {\bfseries{Q}}uery {\bfseries{L}}anguage - a standardized language to interface with databases\\ \hline
    DDL & {\bfseries{D}}ata {\bfseries{D}}efinition {\bfseries{L}}anguage - tells a database how it data will be stored or organized. \\ \hline
    DML & {\bfseries{D}}ata {\bfseries{M}}anipulation {\bfseries{L}}anguage - tells the database how to operate the data. \\ \hline
    Database Normalization & Structuring a relational database, reduce data duplication and improve data integrity. \\ \hline
\end{tabular}



\subsection{Action Points - Further Reading/Enquiry}


\begin{adjustbox}{max width=\textwidth}
    \begin{tabular}{c|l|l|c|l}
        Action Point & To-do description & Assigend to & Target date & Comment/Status \\
        \hline
        1 & Verify/enroll to Teams channel membership for the course & N/A & ASAP & Assigned \\ \hline
        2 & Setup a home-lab with SQLite & N/A & ASAP & Assigned \\ \hline
        3 & Look up and learn SQL (DDL, DML etc) & N/A & ASAP & Assigned \\ \hline
        4 & Look up and learn UML & N/A & ASAP & Assigned \\ \hline
    \end{tabular}
\end{adjustbox}


\subsection{Other source materials} \label{subsec:source}


\begin{tabular}{p{20mm} | p{120mm}}
    {\bfseries{Resource Type}} & {\bfseries{Source description, Book title, URL, etc.}}\\
    \hline
    Wikipedia - 3rd Normal Form & \url{https://en.wikipedia.org/wiki/Third_normal_form}\\ \hline
    Wikipedia - SQLite & \url{https://en.wikipedia.org/wiki/SQLite}\\ \hline
    Youtube - SQLite & \url{https://www.youtube.com/watch?v=byHcYRpMgI4}\\ \hline
    YouTube - SQLite usecases & \url{https://www.youtube.com/watch?v=Jib2AmRb_rk}\\ \hline
    SQLite - Official & \url{https://sqlite.org/index.html}\\ \hline
\end{tabular}


\subsection{Issues Noted and Area of Improvements}

\begin{tabular}{c|l}
    Issue number & Issue description / Area of Improvement\\
    \hline
    1 & N/A\\ \hline
    2 & N/A\\ \hline
    
\end{tabular}