% This is a section template for reflective journal

% ----> Header informaton section - Start
\section{Lesson 1:Introduction} % Replace the double angle brackets with the lesson number and name.

Original URL: << link to the original blog post on the Moodle blog>> % Replace the double angle brackets with the URL to the reflective journal blog.
% <---- Header informaton - End



% ----> Main reflective subsection - Start
% ----| Write reflective thoughts on the topic in general. |----
\subsection{Reflection on the days lecture and tutorial}

<<Add critical reflective thoughts about your learning experiences. Delete this text>>

{\bfseries{Lesson/Topic goals/Study Plan:}}
\begin{itemize}
    \item Why...
    \item What...
    \item How...
\end{itemize}
% ----> Main reflective subsection - End


% ----> Main reflective subsection - Start
% ----| Write reflective thoughts on a specific reflective topic. |----
\subsection{Reflection Topics}

<<When there are guided topics add them as sub-headings and include your critical discussion of the topic(s)>>

\blindtext[3]  % Example text, comment out or remove this line.

\subsubsection{Provided topic 1}

<<topic 1 example reflective discussion>>

\blindtext[3] % Example text, comment out or remove this line.

\subsubsection{Provided topic 2}

<<topic 2 example reflective discussion >>


\insertcode{tex/TestCode.py}{This is a Python code} % Example of how to insert code with modified formatting. Delete or comment out.

% ----> Main reflective subsection - End



% ----> Main reflective subsection - Start
% ----| Write Itemized notes regarding the lesson topic. |----
\subsection{Key Take-Away}

This subsection outlines key information from the day's lesson in bullet points.

\begin{enumerate}
    \item Main Item 1.
    \begin{enumerate}
        \item sub item 1 1
        \item sub item 1 2
        \item sub item 1 3
        \item sub item 1 4
        \item sub item 1 5
        \item sub item 1 6
    \end{enumerate}
    \item Main Item 2
    \begin{enumerate}
        \item sub item 2 1
        \item sub item 2 2
        \item sub item 2 3
        \item sub item 2 4
    \end{enumerate}
    \item ..
    \item .
\end{enumerate}


\subsection{Lessons Learned}

{\emph{Below text are example only!!!}}


This subsection summarizes the day's lesson topic.

{\bfseries{Databases}}


Databases is an element of modern information infrastructure.


Organized data/information by following examples.


\begin{itemize}
    \item Catagorization
    \item Quantify
    \item Itemization
    \item Relation etc.
\end{itemize}


Redundant data are problematic:
\begin{itemize}
    \item Storage, takes space
    \item Overhead, when updating
    \item Itegrity, data consistency
\end{itemize}


{\bfseries{Glossary:}}


% \begin{tabular}{@{}c|c @{}}
\begin{tabular}{p{40mm} | p{100mm}}
    Word/expression & Elaboration / Comment \\ \hline
    3rd Normal Form & A database schema design see "Other source material" table in section \ref{subsec:source}. Ensures update and insert integrity to the database.\\ \hline
\end{tabular}



\subsection{Action Points - Further Reading/Enquiry}

\begin{adjustbox}{max width=\textwidth}
\begin{tabular}{c|l|l|c|l}
    Action Point & To-do description & Assigend to & Target date & Comment/Status \\
    \hline
    1 & Create home-lavb environment for the course & Self & ASAP & Assigned \\
\end{tabular}
\end{adjustbox}

\subsection{Other source materials} \label{subsec:source}


\begin{tabular}{c|c}
    Resource Type & Source description, Book title, URL, etc.\\
    \hline
    Wikipedia & \url{https://en.wikipedia.org/wiki/Third_normal_form}\\ \hline
    Row 2 & Cell 2\\ \hline
\end{tabular}



\url{www.bing.com}\\
\url{www.wikipedia.com}\\
\url{www.wolframalpha.com}\\

\subsection{Issues Noted and Area of Improvements}

\begin{tabular}{c|l}
    Issue number & Issue description / Area of Improvement\\
    \hline
    1 & AOI description. Lorem ipsum\\ \hline
    2 & Issue description. Lorem ipsum\\
    
\end{tabular}